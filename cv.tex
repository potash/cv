%________________________________________________________________________________________
% @info     Based on Latex Resume Template by Chris Paciorek 
%           http://www.biostat.harvard.edu/~paciorek/

%________________________________________________________________________________________
\documentclass[margin,line]{resume}
\usepackage{amsfonts}
\usepackage{hyperref}
\begin{document}
\name{\Large Eric Potash}
\address{epotash@uchicago.edu / \href{http://www.k2co3.net}{k2co3.net} / \href{https://github.com/potash}{github.com/potash}}
\begin{resume}
\section{\mysidestyle Experience}
	{\bf University of Chicago}\hfill{2017--Now}\\
	Postdoctoral Scholar (Advisor: Dan Black), Harris School of Public Policy
%        \begin{itemize}
%		\item Policy analysis of lead poisoning prevention programs for the Chicago\\Department of Public Health
%	\end{itemize}

    %____________________________________________________________________________________
    % Education
    \section{\mysidestyle Education}
	
	{\bf Northwestern University} \hfill {2009--2014} \\%
	Ph.D. Mathematics (Advisor: Steve Zelditch) \\
	Dissertation: Euclidean Embeddings and Riemannian Bergman Metrics

	{\bf Columbia University} \hfill {2005--2009} \\%
    B.A. Mathematics with Honors, Columbia College Class of 2009 \\
	Thesis: An Application of Poincar\'e's Fundamental Polyhedron Theorem

	
    %____________________________________________________________________________________
    % Publications
	\section{\mysidestyle Publications}
        {\bf Randomization Bias in Field Trials to Evaluate Targeting Methods}\\
        \textit{Economics Letters}, Volume 167, June 2018, Pages 131--135.\\
        \textbf{E. Potash}

	{\bf Predictive Modeling for Public Health: Childhood Lead Poisoning} \\
        \textit{21st ACM SIGKDD Proceedings} \\
        \textbf{E. Potash}, J. Brew, A. Loewi, S. Majumdar, A. Reece, J. Walsh, E. Rozier, E. Jorgensen, R. Mansour, R. Ghani

	{\bf Euclidean Embeddings and Riemannian Bergman Metrics} \\
        \textit{The Journal of Geometric Analysis}, January 2016, Volume 26, Issue 1, pp 499-528\\
        \textbf{E. Potash}

	%{\bf An Asymptotic for the Representation of Integers as Sums of\\Triangular Numbers} \\
        %\textit{Involve}, 2008, no. 1, p. 111-121. (with A. Atanasov, R. Bellovin,\\ I. Loughman-Pawelko and L. Peskin)
        %
        \section{\mysidestyle Working Papers}
        {\bf Prediction-Based Decisions and Fairness: A Catalogue of Choices,\\Assumptions, and Definitions}\\
        S. Mitchell, \textbf{E. Potash}, S. Barocas \\

        \section{\mysidestyle Work in Progress}
        {\bf Validation of a Machine Learning Prediction Model of Elevated Blood Lead Levels}\\
        \textbf{E. Potash}, R. Ghani, E. Jorgensen, C. Lohff, N. Prachand, R. Mansour

        {\bf The Effect of Early Interventions on Childhood Lead Exposure}\\
        with Emile Jorgensen

        \section{\mysidestyle Other Writing}
        {\bf Why It's So Hard to Find Out Where the Candidates Stand} \\
        \textit{Washington Monthly}, November 2016

	\section{\mysidestyle Invited Talks}
        
	{\bf Environmental Policy Institute at Chicago (EPIC) Workshop} \\
        Can Health Departments Prevent Childhood Lead Poisoning?, 5/15/2018

	{\bf EPA Research and Development ``Science at Work'' Seminar} \\
        Proactive Lead Investigations, 4/12/2017

	{\bf City Bureau Public Forum} \\
        Lead Poisoning Panel Speaker, 3/13/2017 

	{\bf American Public Health Association Annual Meeting} \\
    Predictive Analytics in Advancing Public Health Session, 11/3/2015 

	{\bf Bloomberg Data for Good Exchange} \\
        Predictive Modeling for Public Health: Childhood Lead Poisoning, 9/30/2015 

    {\bf ACM Knowledge Discovery and Data Mining (KDD) Annual Conference} \\
        Predictive Modeling for Public Health: Childhood Lead Poisoning, 8/12/2015

	\section{\mysidestyle Grants}
        {\bf Collecting and Sharing Information across Sectors in Chicago and Illinois\\ to Identify Children at Risk for Lead Poisoning}. Robert Wood Johnson\\ Foundation. With Rayid Ghani, Raed Mansour, Matthew Roberts, John DiCello,\\ Tom Schenk,  Illinois Department of Human Services, and Alliance of Chicago.\\ Grant ID 73354. \$200,000. \\

        \section{\mysidestyle Industry Experience}
	{\bf University of Chicago}\hfill{2014--2017}\\
	Research Professional II, Center for Data Science and Public Policy

%	\begin{itemize}
%		\item Predictive modeling of lead poisoning for the Chicago\\Department of Public Health
%		\item Predictive modeling of hazardous waste violations for the U.S. Environmental\\ Protection Agency and New York State Department of Environmental\\ Conservation
%	\end{itemize}
        {\bf Eric and Wendy Schmidt Data Science for Social Good}\hfill{Summer 2016}\\
	Technical Mentor

%	\begin{itemize}
%		\item Mentored graduate students in analysis and development of data\\science solutions for public policy problems.
%	\end{itemize}
%        {\bf University of Chicago}\hfill{Winter 2016}\\
%        Lecturer, Harris School of Public Policy
%	\begin{itemize}
%		\item Computation for Public Policy graduate course
%	\end{itemize}
	{\bf Open Energy Efficiency Meter} (\href{http://www.openeemeter.org/}{openeemeter.org}) \hfill {2015} \\
	Data Scientist

%	Statistical learning of residential energy consumption baselining and forecasting.
%	{\bf Eric and Wendy Schmidt Data Science for Social Good}\hfill{Summer 2014}\\
%	Summer Fellow
%	\begin{itemize}
%		\item Modeling maternal health outcomes for the government of Mexico.
%		\item Electricity load disaggregation for Pecan Street Research Institute.
%	\end{itemize}
	{\bf Oroeco} (\href{http://www.oroeco.com}{oroeco.com}) \hfill {2014} \\
	Scientific Software Engineer %	Collecting data and building carbon footprint models and visualizations. 
        \section{\mysidestyle Teaching}
        {\bf University of Chicago}\hfill{2016}\\
        Introduction to Program Evaluation (Spring 2019)\\Introduction to Programming for Public Policy (Spring 2018, 2016)\\\\
%        Lecturer, Harris School of Public Policy
%	\begin{itemize}
%		\item Computation for Public Policy graduate course
%	\end{itemize}
	{\bf Northwestern University} \hfill {2008--2013}\\
	Assistant: Probability \& Stochastic Processes, Mechanics, Real Analysis%, Hyperbolic Geometry, Calculus
	
	%{\bf Federal Reserve Bank of New York} \hfill {2006}\\%
	%Technology Summer Analyst; New York, NY


	%\section{\mysidestyle Volunteer} 
	%{\bf Habitat 2030} \hfill {2013--Now} \\
	%Chicago-area ecological habitat restoration and stewardship.

	%{\bf Open Source Ecology} \hfill {2011--Now} \\
	%Building and documenting an open source compressed earth \href{http://opensourceecology.org/portfolio/ceb-press/}{brick press} and \\ sustainable, modular, low-cost \href{http://opensourceecology.org/portfolio/microhouse/}{house}.

	%{\bf Chicago Botanic Garden} \hfill {2010} \\
	%Helped maintain the Japanese gardens.
    %____________________________________________________________________________________
    % Skills
    \section{\mysidestyle Skills}
		Python (numpy, scipy, pandas, sklearn, matplotlib) \\
                R (dplyr, Stan, ggplot2) \\
                SQL (PostgreSQL), Java, JavaScript (D3.js), Ruby (on Rails)\\
                Geospatial (PostGIS, GDAL, OpenStreetMap, Mapnik, QGIS, Leaflet)\\
                git, bash, GNU/Linux, \LaTeX\\
		%Probability, Causal Inference, Differential Geometry, Partial Differential Equations\\
                Fluent in Russian
	\section{\mysidestyle References}
		\begin{itemize}

        \item Dan Black, danblack@uchicago.edu \\
                Professor, Harris School of Public Policy, University of Chicago
	\item Matt Gee, mattgee@gmail.com \\
		Research Fellow, Urban Center for Computation and Data 
	\item Emile Jorgensen, Emile.Jorgensen@cityofchicago.org \\
		Epidemiologist, Chicago Department of Public Health
	\item Steve Zelditch, s-zelditch@northwestern.edu \\
		Wayne and Elizabeth Jones Professor of Mathematics, Northwestern University 
		\end{itemize}
%________________________________________________________________________________________
\end{resume}
\end{document}

%________________________________________________________________________________________
% EOF

